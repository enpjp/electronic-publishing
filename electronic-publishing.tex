
\documentclass{article}
\usepackage[a-1b]{pdfx}
\usepackage{arxiv}

\usepackage[utf8]{inputenc} % allow utf-8 input
\usepackage[T1]{fontenc}    % use 8-bit T1 fonts
\usepackage{lmodern}        % https://github.com/rstudio/rticles/issues/343
%\usepackage{hyperref}       % hyperlinks
\usepackage{url}            % simple URL typesetting
\usepackage{booktabs}       % professional-quality tables
\usepackage{amsfonts}       % blackboard math symbols
\usepackage{nicefrac}       % compact symbols for 1/2, etc.
\usepackage{microtype}      % microtypography
\usepackage{graphicx}
\usepackage[style=british]{csquotes}
\hypersetup{hidelinks}
\title{\myTitleMainTitle}

\author{
    Dr Paul J. Palmer
    \thanks{Written to fulfil an action of the LES AGM.}
   \\
    Leicestershire Entomological Society \\
  \texttt{\href{mailto:palmerpjp@gmail.com}{\nolinkurl{palmerpjp@gmail.com}}} 
} 



% tightlist command for lists without linebreak
\providecommand{\tightlist}{%
  \setlength{\itemsep}{0pt}\setlength{\parskip}{0pt}}

% From pandoc table feature
\usepackage{longtable,booktabs,array}
\usepackage{calc} % for calculating minipage widths
% Correct order of tables after \paragraph or \subparagraph
\usepackage{etoolbox}
\makeatletter
\patchcmd\longtable{\par}{\if@noskipsec\mbox{}\fi\par}{}{}
\makeatother
% Allow footnotes in longtable head/foot
\IfFileExists{footnotehyper.sty}{\usepackage{footnotehyper}}{\usepackage{footnote}}
\makesavenoteenv{longtable}

% Pandoc citation processing
\newlength{\cslhangindent}
\setlength{\cslhangindent}{1.5em}
\newlength{\csllabelwidth}
\setlength{\csllabelwidth}{3em}
\newlength{\cslentryspacingunit} % times entry-spacing
\setlength{\cslentryspacingunit}{\parskip}
% for Pandoc 2.8 to 2.10.1
\newenvironment{cslreferences}%
  {}%
  {\par}
% For Pandoc 2.11+
\newenvironment{CSLReferences}[2] % #1 hanging-ident, #2 entry spacing
 {% don't indent paragraphs
  \setlength{\parindent}{0pt}
  % turn on hanging indent if param 1 is 1
  \ifodd #1
  \let\oldpar\par
  \def\par{\hangindent=\cslhangindent\oldpar}
  \fi
  % set entry spacing
  \setlength{\parskip}{#2\cslentryspacingunit}
 }%
 {}
\usepackage{calc}
\newcommand{\CSLBlock}[1]{#1\hfill\break}
\newcommand{\CSLLeftMargin}[1]{\parbox[t]{\csllabelwidth}{#1}}
\newcommand{\CSLRightInline}[1]{\parbox[t]{\linewidth - \csllabelwidth}{#1}\break}
\newcommand{\CSLIndent}[1]{\hspace{\cslhangindent}#1}

%\usepackage{booktabs}
%\usepackage{longtable}
%\usepackage{morefloats}
%\extrafloats{100}
%\date{August 25, 2021}
%\renewcommand{\today}{September 5, 2021}

% Set the copyright footer
%\lfoot{\copyright 2021 P.J. Palmer  P.M. Leonard}

% Some figure placement options.
%\usepackage[figuresonly,nomarkers,fighead, figlist]{endfloat}
%\usepackage[figuresonly,nomarkers,nolists]{endfloat}
% Put multiple figures per page
%\renewcommand{\efloatseparator}{\mbox{}}
\usepackage{flafter}
\usepackage{booktabs}
\usepackage{longtable}
\usepackage{array}
\usepackage{multirow}
\usepackage{wrapfig}
\usepackage{float}
\usepackage{colortbl}
\usepackage{pdflscape}
\usepackage{tabu}
\usepackage{threeparttable}
\usepackage{threeparttablex}
\usepackage[normalem]{ulem}
\usepackage{makecell}
\usepackage{xcolor}
\begin{document}
\maketitle


\begin{abstract}
\myAbstract
\end{abstract}


\hypertarget{introduction}{%
\section{Introduction}\label{introduction}}

Electronic publication offers opportunities for producing of authoritative reports with citable references other than with the ISBN associated with traditional publication through printed documents. The use of a DOI (Digital Object Identifier) is well known in the academic community, as a means to permanently reference online versions of published papers in addition to the ISBN or ISSN associated with the printed journal or conference proceedings. However, a DOI may also be used to identify many other forms of online resource including data.

The intention of both systems is to provide a unique reference that may be freely shared to identify a resource such as a book, report or database. But this does not imply that full access is free of charge.  Several methods are reviewed here that should be accessible to small societies that seek to publish informative articles for a community of interest.

\section{Amazon Kindle Publishing Direct (KDP)}

Amazon provides a complete publication and marketing package that allows the authors sell their work online. Free publication is not possible as Amazon takes a proportion of the sales to fund the service. You can use your own ISBN or Amazon will assign one for you free of charge. Both the printed and kindle versions may be managed from the same account.

The service is easy to use for anyone with reasonable desk top publishing (DTP) skills and automated checking helps to eliminate problems as they arise. Word templates are provided, but you are better off using specialist DTP software such as Latex or Scribus if your work includes many images.

The quality of the finished product is excellent and turnaround of orders very fast. PJP published. If the publication is updated, you can choose whether to produce a new edition or just to update the electronic master.

An example  is here:

\url{https://www.amazon.co.uk/How-Build-Your-Moth-Trap/dp/1981158197}

Do be prepared for unfavourable reviews as there are people who mark the grammar and typos as a hobby.

\section{PDF with DOI}

The internet has brought many changes to academic publication and a trend towards Open Access documents that may easily be shared. Many authors publish PDF versions of their work outside the formal journal infrastructure using \enquote{pre-print} servers. The name is a little misleading since some documents may never be intended for formal publication, but their contents may help support other more substantive work.

These services are free and suitable for publication of authoritative reports from individuals and small societies. One such service is zenodo.org.  It is important to realise that managing changes is a feature of the service which is why it is different from posting an uncontrolled copy on a website. 

In this example PJP has set up a community to share reports on VC55 spiders. \\
See: \url{https://zenodo.org/communities/vc55-spiders/} 

Palmer, Paul J. (2022). Report on VC55 Spider Status In The Rutland Water Area. Zenodo. \url{https://doi.org/10.5281/zenodo.6048792}

This report is likely to be updated over time so the DOI \\ \url{https://doi.org/10.5281/zenodo.6048791} \\will always point towards the latest version. However, each version also has its own DOI so \\
\url{https://doi.org/10.5281/zenodo.6048792} \\
 will always point towards version 1.
 
 
 Note that the report also includes a CSV version of the checklist as an example of how online publication may included mixed resources.

Zenodo accounts are free and only one person need have an account to upload publications. However, using multiple trusted people to manage a community would be wise to ensure continuity of access for all time.

Note that the reference is also easily found using a search engine. Try: \enquote{Report on VC55 Spider Status}.

\section{Producing PDF Documents}

Microsoft Word, although popular, is not really suitable for producing publication ready PDF documents. While many people manage to get reasonable looking results, underneath the surface are many problems with fonts and images.

The tool that stands out for producing high quality PDF documents is Latex. However, there is a steep leaning curve which many will find off-putting.  It has the huge advantage of completely separating the content from output by using style-sheets, enabling the author to concentrate on writing, rather than layout. This report was created in this manner.

Overleaf (see: \url{https://www.overleaf.com/})  provides an excellent service that requires no local software with many useful guides. Multiple people can work on the same document, and as long as you have one person who knows Latex, everyone else can just type content.

\subsection{PDF\textbackslash A}

The PDF\textbackslash A format has been developed for long term preservation  as it embeds all the information required to reproduce the original document, plus supporting metadata. Ensuring full compliance to the PDF\textbackslash standard requires access to external validation tools, but nominal compliance is achieved in Latex by adding the following command to the document head of this report:
 \begin{verbatim}
	\usepackage[a-1b]{pdfx}
\end{verbatim}

The following metadata was also included:
 \begin{verbatim}
\Title{Methods of Electronic Publication Suitable for Small Societies}
\Author{P.J. Palmer}
\Language{en-GB}
\Keywords{Electronic publication\sep Small societies}
\Publisher{Leicestershire Entomological Society}
\end{verbatim}
 
 \section{Concluding Comments}
 
Electronic publication does not supersede the role of printed media, but it does provide new routes to disseminate knowledge in a way that can be cited in other work. There are no reasons why printed reports with an ISBN should not be assigned a DOI for freely available PDF versions.

Each report produced with a DOI will be available for the foreseeable future, so it is reasonable to expect that more effort is required to ensure long term accessibility of the PDF and associated resources.


\bibliographystyle{unsrt}
\bibliography{references.bib}


\end{document}
